\documentclass[a4paper]{article}

%% Language and font encodings
\usepackage[portuges]{babel}
\usepackage{fontspec}

%% Sets page size and margins
\usepackage[a4paper,top=3cm,bottom=2cm,left=3cm,right=3cm,marginparwidth=1.75cm]{geometry}

%% Useful packages
\usepackage{amsmath,amsthm,amssymb,amsfonts}
\usepackage{graphicx}
\usepackage[colorinlistoftodos]{todonotes}
\usepackage[colorlinks=true, allcolors=blue]{hyperref}

\newcommand{\R}{\mathbb{R}}
\newcommand{\N}{\mathbb{N}}
\newcommand{\Z}{\mathbb{Z}}
\providecommand{\C}{\mathbb{C}}

\theoremstyle{definition}
\newtheorem{defin}{Definição}

\theoremstyle{plain}
\newtheorem{theorem}[defin]{Teorema}
\newtheorem{corollary}[defin]{Corolário}



\title{Useful latex equation I need}

\author{Thai Thien\\
        Meow\\
}

\date{24/06/2020}

\begin{document}
\maketitle

%\begin{abstract}
%Your abstract.
%\end{abstract}

\section{Define count counting problem}

\[ C(I_i) = \sum_{P \in P_{i}} P \]

\[ C(I_i) = n \]


\[ C(I_i) = \int_{p \in I_{i}} F_{i}(p) \]



%%%%%%%%%%%%%%
%%%%%%%%%%%%


\section{Integral of 2D Gaussian}

ref: https://www.youtube.com/watch?v=jNrxkMONL5A

\[ \forall p \in I_{i}, \quad F_{i}^{0}(p)=\sum_{P \in \mathbf{P}_{i}} \mathcal{N}\left(p ; P, \sigma^{2} \mathbf{1}_{2 \times 2}\right) \]


\[ \int_{-\infty}^{\infty}   F_{i}^{0}(p) dp = \int_{-\infty}^{\infty} \sum_{P \in \mathbf{P}_{i}} \mathcal{N}\left(p ; P, \sigma^{2} \mathbf{1}_{2 \times 2}\right) dp \]
\[ \int_{-\infty}^{\infty}\mathcal{N}\left(p ; P, \sigma^{2} \mathbf{1}_{2 \times 2}\right) dp = 1 \]
We have Gaussian function with mean = 0 and std.
\[ f(x) = \frac{1}{\sqrt{2\pi\sigma^2}} e^\frac{-x^2}{2\sigma^2} \]


In 2D

\[ g(x, y) = f(x)f(y) = \frac{1}{2\pi\sigma^2} e^\frac{-(x^2+y^2)}{2\sigma^2} \]



\[ Let A = \int_{x=-\infty}^{\infty}\int_{y=-\infty}^{\infty} g(x,y)dydx = \frac{1}{2\pi\sigma^2} \int_{x=-\infty}^{\infty}\int_{y=-\infty}^{\infty} e^\frac{-(x^2+y^2)}{2\sigma^2} dydx\]

\[ Let B =  \int_{x=-\infty}^{x=\infty}\int_{y=-\infty}^{y=\infty} e^\frac{-(x^2+y^2)}{2\sigma^2} dydx\]

If we want to prove A = 1, so we prove B = $2\pi\sigma^2$

Convert to Polar coordinate 

\[ r = \sqrt{x^2 + y^2} \]

\[ B =  \int_{\theta=0}^{\theta=2\pi}\int_{r=0}^{y=\infty} e ^ \frac{-r^2}{2\sigma^2}rdrd\theta\]

\[ B =  2\pi\sigma^2 [-e^\frac{-1r^2}{2\sigma^2}]_0^{\infty} \]

\[ B =  2\pi\sigma^2 [0 - (-1)] \]

\[ B =  2\pi\sigma^2  \]

\section{Without inf}


\[ \forall p \in I_{i}, \quad F_{i}^{0}(p)=\sum_{P \in \mathbf{P}_{i}} \mathcal{N}\left(p ; P, \sigma^{2} \mathbf{1}_{2 \times 2}\right) \]


\[ \int   F_{i}^{0}(p) dp = \int \sum_{P \in \mathbf{P}_{i}} \mathcal{N}\left(p ; P, \sigma^{2} \mathbf{1}_{2 \times 2}\right) dp \]

\[ \int\mathcal{N}\left(p ; P, \sigma^{2} \mathbf{1}_{2 \times 2}\right) dp = 1 \]


We have Gaussian function with mean = 0 and std.
\[ f(x) = \frac{1}{\sqrt{2\pi\sigma^2}} e^\frac{-x^2}{2\sigma^2} \]


In 2D

\[ g(x, y) = f(x)f(y) = \frac{1}{2\pi\sigma^2} e^\frac{-(x^2+y^2)}{2\sigma^2} \]



\[ Let A = \frac{1}{2\pi\sigma^2} \int\int e^\frac{-(x^2+y^2)}{2\sigma^2} dydx\]

\[ Let B =  \int\int e^\frac{-(x^2+y^2)}{2\sigma^2} dydx\]

If we want to prove A = 1, so we prove B = $2\pi\sigma^2$

Convert to Polar coordinate

\[ B =  \int_{\theta=0}^{2\pi}\int_{r=0}^{\infty} e ^ \frac{-r^2}{2\sigma^2}rdrd\theta\]

\[ B =  2\pi\sigma^2 [-e^\frac{-1r^2}{2\sigma^2}]_0^{\infty} \]

\[ B =  2\pi\sigma^2 [0 - (-1)] \]

\[ B =  2\pi\sigma^2  \]

\end{document}